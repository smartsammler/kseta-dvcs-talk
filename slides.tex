\documentclass[18pt,mathserif]{beamer}

%{{{ Settings
\usetheme[english, titlepage0]{KIT}

%{{{ Packages
\usepackage{tikz}
%\usepackage{xltxtra}
\usepackage{csquotes}
\usepackage{caption}
\usepackage{hyperref}
\usepackage[framemethod=tikz]{mdframed}
%}}}
%{{{ Fonts and colors
%\setsansfont[Mapping=tex-text,
%             BoldFont={* Semibold}]{Source Sans Pro}
%\setmonofont[UprightFeatures={LetterSpace=-0.5}]{Inconsolata}

% TomorrowTheme: https://github.com/chriskempson/tomorrow-theme/blob/master/vim/colors/Tomorrow.vim
\definecolor{syntax-comment}{HTML}{8e908c}
\definecolor{syntax-keyword}{HTML}{8959a8}
\definecolor{syntax-number}{HTML}{eab700}
\definecolor{syntax-green}{HTML}{718c00}
\definecolor{syntax-red}{HTML}{c82829}

\setbeamercolor{section in toc}{fg=black!80}
%}}}
%{{{ TikZ styles
\usetikzlibrary{arrows, backgrounds, calc, trees, shapes, snakes}

\tikzset{
  commit/.style={
    circle, draw, color=black!75, fill=syntax-keyword!75,
    line width=1pt, minimum height=12pt
  },
  connection/.style={
    draw, ->, >=stealth, color=black!75, line width=.75pt
  }
}

\pgfdeclareimage[width=.36cm]{butthurt}{images/butthurt}

\newcommand\score[2]{
  \pgfmathsetmacro\pgfxa{#1+1}
  \begin{tikzpicture}[baseline=-3pt]
    \foreach \i in {1,...,#2} {
      \pgfmathparse{(\i<=#1?"1.0":"0.3")}
      \edef\starcolor{\pgfmathresult}
      \node[name=butthurt\i, opacity=\starcolor, inner sep=0pt] at (\i*3ex,0) {\pgfuseimage{butthurt}};
    }
  \end{tikzpicture}
}

\makeatletter
\tikzset{
  circle split part fill/.style  args={#1,#2}{%
    alias=tmp@name, % Jake's idea !!
    postaction={%
      insert path={
        \pgfextra{%
          \pgfpointdiff{\pgfpointanchor{\pgf@node@name}{center}}%
                       {\pgfpointanchor{\pgf@node@name}{east}}%
          \pgfmathsetmacro\insiderad{\pgf@x}
          \fill[#1] (\pgf@node@name.base) ([xshift=-\pgflinewidth]\pgf@node@name.east) arc
                            (0:180:\insiderad-\pgflinewidth)--cycle;
          \fill[#2] (\pgf@node@name.base) ([xshift=\pgflinewidth]\pgf@node@name.west)  arc
                            (180:360:\insiderad-\pgflinewidth)--cycle;
        }
      }
    }
  }
}
\makeatother
%}}}
%{{{ Environments
\makeatletter
\newenvironment{shell}[1][\linewidth]
  {\begin{mdframed}[
  skipabove=\topsep,
  skipbelow=\topsep,
  font=\ttfamily,
  linecolor=black!9,
  backgroundcolor=black!9,
  innertopmargin=6pt,
  innerbottommargin=6pt,
  innerleftmargin=6pt,
  innerrightmargin=6pt,
  userdefinedwidth=#1]}
  {\end{mdframed}}
\makeatother
%}}}
%{{{ Meta
\title{Git -- source code management}
\author{Julian Gethmann}
\subtitle{based on last years workshop by Eileen K\"uhn, David Kunz, Sarah M\"uller, Robin Roth}
\institute{KSETA Doktorandenworkshop, 7.7.2015}
\date{Jul.~7\textsuperscript{th} 2015}

\graphicspath{{../../common/}}

\TitleImage[width=4cm]{images/Git-Logo-2Color.eps}
%}}}

%}}}

\begin{document}
\maketitle

\section{Motivation}

\begin{frame}{Plead guilty!}
  It's easy to copy digital content, so why not re-create it over and over
  again?

  \begin{columns}[onlytextwidth]
    \visible<2->{
      \column{0.5\textwidth}
        \begin{figure}
          \centering
          \includegraphics[width=4.5cm]{images/mrcs.png}
          \caption*{\enquote{One of these folders \emph{must} contain the latest
          version \ldots}}
         \end{figure}
    }

    \visible<3->{
      \column{0.5\textwidth}
         \begin{figure}
          \centering
          \includegraphics[width=4.5cm]{images/reports.png}
          \caption*{\enquote{Here is the latest version of the
          proposal/paper/report.} --- \enquote{Thanks.}}
        \end{figure}
    }
  \end{columns}
\end{frame}
\begin{frame}{Obvious disadvantages}
  \begin{itemize}
    \ item No meta data about \emph{what} was changed \emph{when} by
      \emph{whom}
    \item You lose track of what's going on
    \item You cannot easily roll-back to a working state
    \item Poor solution for collaboration or work on different parts at different places
  \end{itemize}
  \begin{itemize}
    \item This talk is a copy of the last years talk and doesn't have those problems eigther.
    \item Why not?
  \end{itemize}
\end{frame}
\begin{frame}{Version control}
	\begin{itemize}
		\item \emph{Track} files
		\item Record (\emph{commit}) changes
		\item Work on different copies 
		\item Share changes with others
		\item Roll-back to an earlier state
		\item Implicit backup

	\end{itemize}
\end{frame}

\begin{frame}{Why Git?}
  \begin{columns}[T,onlytextwidth]
    \column{0.7\textwidth}
        \begin{itemize}
          \item De-facto standard for open source software
          \item Probably the fastest version control system out there
	  \item Easy to learn, though feature rich
	  \item Works well both with central and distributed repositories
          \item GitHub: web based collaboration platform
        \end{itemize}
    \column{0.3\textwidth}
      \centering
      \includegraphics[width=4.0cm]{images/octocat.png}
  \end{columns}
\end{frame}
 
\section{Hands-on introduction}

\begin{frame}{}
  \begin{center}
    \huge\bfseries
    \textcolor{KITblack}{Git Basics}
  \end{center}
\end{frame}

\begin{frame}{Configuration}
  \begin{itemize}
    \item{Tell git who you are}
      \begin{shell}
        \$ git config --global user.name <name>\\
        \$ git config --global user.email <email>
      \end{shell}
    \item{Configure auto correct for git commands}
      \begin{shell}
        \$ git config --global help.autocorrect 1
      \end{shell}
    \item{Use colors to show git information}
      \begin{shell}
        \$ git config --global color.ui auto
      \end{shell}
  \end{itemize}
\end{frame}

\begin{frame}{Single User Workflow}
  \begin{enumerate}
    \item Create a repository and a branch ``master''
      \begin{shell}
        \$ git init
      \end{shell}
    \item Create a commit
      \begin{enumerate}
        \item Add something to the commit
          \begin{shell}
            \$ git add README.txt
          \end{shell}
        \item Perform the commit
          \begin{shell}
            \$ git commit -m "Add a README file"
          \end{shell}
      \end{enumerate}
  \end{enumerate}
	\begin{center}
	\includegraphics[width=.8\linewidth]{pics/init0.png}
	\end{center}
\end{frame}

% Live Demo 1

\begin{frame}{Commits}
  Everytime you make a change, you create a \alert{commit} containing:
  \begin{itemize}
    \item added/removed lines in files
    \item a comment summarizing what was changed
    \item an author
    \item a date
    \item a checksum (SHA-hash) to identify the commit
    \item a reference to the previous state of your files (parent(s))
  \end{itemize}
\end{frame}

% \begin{frame}{Commit Graphs}


%   \begin{itemize}
%     \item The first commit does not have a parent. It creates a \alert{repository}.
%     \item Multiple commits can have the same parent (e.g. when coworkers work on the same code).
%     \item Commits can have multiple parents if the two diverging changes are merged.
%     \item Commits can have \alert{labels}. Labels can be e.g. branch names version tags.
%   \end{itemize}
% \end{frame}


\begin{frame}[t]{Single User Workflow}
  \begin{enumerate}
    \item<alert@1> Change something, and inspect the difference to the last commit
      \begin{shell}
        \$ vi README.txt\\
        \$ git diff\\
      \end{shell}
    \item Create a commit (as before)
      \begin{enumerate}
        \item<alert@2> Add some changes to the commit
          \begin{shell}
            \$ git add README.txt
          \end{shell}
        \item<alert@3> Perform the commit
          \begin{shell}
            \$ git commit %-m "Add project description"
          \end{shell}
      \end{enumerate}
  \end{enumerate}
  \begin{overprint}
		\includegraphics<1>[width=.8\linewidth]{pics/commit0.png}
		\includegraphics<2>[width=.8\linewidth]{pics/commit1.png}
		\includegraphics<3>[width=.8\linewidth]{pics/commit2.png}
    \end{overprint}
\end{frame}

\begin{frame}{How to commit}
  \begin{columns}[t]
    \column{50mm}
  \begin{itemize}
    \item Small logical units
    \item Several times an hour
			%TODO: add pic
    \item Check the status before committing
      \begin{shell}
	\$ git status
      \end{shell}
    \item Write descriptive commit messages and keep 50/72 limits
		\item[$\Rightarrow$] Allows you to retrace your steps
   \end{itemize}
    \column{50mm}
    \includegraphics[width=\textwidt]{images/18333fig0106-tn.png}
    \url{http://git-scm.com/}
  \end{columns}
 \end{frame}

\begin{frame}{Branching}
  \begin{itemize}
		\item Keep master branch free from ``questionable'' code
			\begin{itemize}
				\item Working on independent features at the same time
				\item Trying incompatible changes
				\item Quick and dirty work without changing the master branch
			\end{itemize}
		\item Cheap, instant and easy
		\item Create and destroy often
		\item Integral part of a typical Git workflow
  \end{itemize}
\end{frame}

\begin{frame}[t]{Branching}
    \begin{itemize}
        \item Create two branches from master
            \begin{shell}
                \alert<1>{\$ git checkout master}\\
                \alert<2>{\$ git checkout -b featureA}\\
                \alert<2>{\$ ...change \& commit something}\\
                \alert<3>{\$ git checkout master}\\
                \alert<4>{\$ git checkout -b featureB}\\
                \alert<5>{\$ ...change \& commit something}
            \end{shell}
    \end{itemize}
    \mbox{%
    \includegraphics<1>[width=\linewidth]{pics/commit2.png}
    \includegraphics<2>[width=\linewidth]{pics/branching1.png}
    \includegraphics<3>[width=\linewidth]{pics/branching2.png}
    \includegraphics<4>[width=\linewidth]{pics/branching3.png}
    \includegraphics<5>[width=\linewidth]{pics/branching4.png}
    }
  \href{http://pcottle.github.io/learnGitBranching/?NODEMO}{pcottle.github.io/learnGitBranching}
\end{frame}

\begin{frame}[t]{Branching}
    \begin{itemize}
        \item Switch back to master branch
            \begin{shell}
                \alert<1>{\$ git checkout master}
            \end{shell}
        \item Merge your changes into master
            \begin{shell}
                \alert<2>{\$ git merge featureA  \# fast forward}\\
                \alert<3>{\$ git merge --no-ff featureA  \# }\\
                \alert<4>{\$ git merge featureB  \# merge}
            \end{shell}
        \item Delete merged branches
            \begin{shell}
                \alert<5>{\$ git branch -d featureA featureB}
            \end{shell}
    \end{itemize}
    \mbox{%
			%TODO: Bilder fuer --no-ff
    \includegraphics<1>[width=\linewidth]{pics/branching5.png}
    \includegraphics<2>[width=\linewidth]{pics/branching6.png}
    \includegraphics<3>[width=\linewidth]{pics/branching9.png}
    \includegraphics<4>[width=\linewidth]{pics/branching7.png}
    \includegraphics<5>[width=\linewidth]{pics/branching8.png}}
\end{frame}

\begin{frame}{Retracing Your Steps}
  \begin{enumerate}
    \item Check the log
      \begin{shell}
        \$ git log  \# copy the SHA-key
	\$ git log --graph --all --format=format:'%C(bold blue)%h%C(reset) - %C(bold green)(%ar)%C(reset) %C(white)%s%C(reset) %C(bold white)— %an%C(reset)%C(bold yellow)%d%C(re    set)' --abbrev-commit --date=relative
	 \$ git log --graph --all --format=format:'%C(bold blue)%h%C(reset) - %C(bold cyan)%aD%C(reset) %C(bold green)(%ar)%C(reset)%C(bold yellow)%d%C(reset)%n''          %C(whi    te)%s%C(reset) %C(bold white)— %an%C(reset)' --abbrev-commit
      \end{shell}
    \item Show changes to current version
      \begin{shell}
				\$ git diff <paste SHA key>
      \end{shell}
    \item Check out old version
      \begin{shell}
        \$ git checkout <paste SHA key>
      \end{shell}
  \end{enumerate}
\end{frame}

% Live Demo 2
% log, branch

\begin{frame}{}
  \begin{center}
    \huge\bfseries
    \textcolor{KITblack}{Collaboration}
  \end{center}
\end{frame}

\begin{frame}
	\frametitle{Working at different places}
	\begin{itemize}
		\item Clone a repository, possible protocols: https, ssh, git, file, \ldots
		 \begin{shell}
			 \$ git clone https://github.com/smartsammler/kseta-dvcs-talk.git
		 \end{shell}
    \item Copies the complete history of all branches to your disk
    \item Stores the cloning source as the \emph{remote} ``origin''
      \begin{shell}
        \$ git remote show\\
        \$ git remote show origin
      \end{shell}
	\end{itemize}
\end{frame}

\begin{frame}{Incorporate Changes of Collaborators}
  \begin{enumerate}
    \item Fetch what others have done
      \begin{shell}
        \$ git fetch
      \end{shell}
      Downloads all commits and labels (e.g. ``origin/master'') from the
      server, but leaves local labels unchanged.
    \item Decide what to do:
      \begin{itemize}
         \item Fast-forward your branch if you did not make changes
         \item Merge a remote branch into your branch
         \item Rebase your branch on top of a remote branch
         \item Cherry-pick a commit from a different branch
      \end{itemize}
  \end{enumerate}
\end{frame}

\begin{frame}{Merge Other Branch Into Yours}
   \begin{itemize}
		\item Trivial merge: fast-forward
		\item Non-trivial: creates new commit which includes both changes
      \begin{shell}
        \$ git merge origin/master
      \end{shell}
		\item Almost always works, but may result in \emph{conflicts} if same lines
      changed in both branch heads
      \begin{shell}
        \$ git config --global mergetool.keepBackup false
        \$ git mergetool
      \end{shell}
    \item Note that you can also do
      \begin{shell}
        \$ git pull
      \end{shell}
      which is the same as a \emph{fetch} and a consecutive \emph{merge}
  \end{itemize}
 \end{frame}

\begin{frame}{Distributing Your Changes}
  \begin{itemize}
    \item Upload changes in your branch ``featureA'' to origin
      \begin{shell}
        \$ git push origin featureA
      \end{shell}
    \item Does not work if featureA is changed on origin, in this case fetch
      and merge first
    \item Does not work if you deleted commits which were on origin, in this
      case don't try to force the update unless you know you are the only one:
      \begin{shell}
        \$ git push -f origin featureA
      \end{shell}
  \end{itemize}
\end{frame}

% \begin{frame}
% 	\frametitle{Group Tasks}
% 	\begin{itemize}
% 		\item \url{https://github.com/ksetagit/groupproject.git}
% 		\item Group tasks in Readme.md
% 	\end{itemize}
% 
% \end{frame}


\begin{frame}{If Something Goes Wrong}
  Things go wrong if changes conflict. You can then:
  \begin{enumerate}
    \item Fix the conflicts, then
      \begin{shell}
        \$ git add <changed files>\\
        \$ git merge % --continue
      \end{shell}
    \item Stop th e operation
      \begin{shell}
        \$ git merge --abort
      \end{shell}
    \item Undo br oken merges:
			% TODO: screenshot fuer reflog
      \begin{shell}
        \$ git reflog\\
        \$ git checkout HEAD@\{1\}
      \end{shell}
  \end{enumerate}
\end{frame}

% Live Demo 3
% merge conflict

% after group session

\begin{frame}{How it works}
	\includegraphics[width=\textwidth]{images/git_vocabulary.png}
\end{frame}

\begin{frame}
	\frametitle{Best Practice Workflow}
	\includegraphics[height=\textheight]{images/git-workflow.png}
	% TODO: Bild abschneiden
\end{frame}

\begin{frame}{Best Practice}
	\begin{itemize}
		\item{Do commit early and often}
		\item{Do not panic (as long as you commited [or even added] your work)}
		\item{Do not change published history (reset/rebase can be evil)}
		\item{Do divide your work into different repositories}
		\item{Do useful commit messages}
		\item{Do keep up to date}
	\end{itemize}
\end{frame}

\begin{frame}{Further reads}
  \begin{columns}[T,onlytextwidth]
    \column{0.7\textwidth}
      \begin{itemize}
        \item \texttt{\$ man git} %\ldots\  just kidding
	\item git help <command>
        \item Free Pro Git book at \url{http://git-scm.com/book}
        \item Different aspects from beginners to pros: \url{http://gitready.com}
        \item Git cheat sheet: \url{http://www.cheat-sheets.org/saved-copy/git-cheat-sheet.pdf}
	\item Interactive git tutorial: \url{https://try.github.io}
	\item Interactive git branching tutorial: \url{https://pcottle.github.io/learnGitBranching/}
	\item Get these slides from: \url{https://github.com/smartsammler/kseta-dvcs-talk}\\
      \end{itemize}
    \column{0.3\textwidth}
      \centering
      \includegraphics[width=1.5cm]{images/pro-git}
  \end{columns}
  \begin{itemize}
    \item Download at \url{http://git-scm.com/downloads}
  \end{itemize}<++>
\end{frame}

\section{Advanced Git Features}

\begin{frame}{}
  \begin{center}
    \huge\bfseries
    \textcolor{KITblack}{Advanced Git Operations}
  \end{center}
\end{frame}

% multiple remotes (pull workflow)

\begin{frame}{Stashing Your Work}
  \begin{itemize}
    \item Get rid of uncommitted changes temporarily
      \begin{shell}
        \$ git stash
      \end{shell}
    \item Resets your working copy to the last committed version $C$
    \item Creates a ``stash commit'' whose parent is $C$
    \item Puts the stash commit on a stack
    \item Top-most stash commit can be applied again using
      \begin{shell}
        \$ git stash pop
      \end{shell}
  \end{itemize}
\end{frame}
\begin{frame}{Rebase Your Branch on Other Branch}
  \begin{itemize}
    \item Most complex operation in git:
      \begin{shell}
        \$ git rebase origin/master
      \end{shell}
    \item Detach a commit from its parent and attach it to another commit
    \item Pre-condition is that changes can be applied to new parent
    \item Pro: Does not result in a merge-commit
    \item Contra: May create cascades of conflicts during rebase
  \end{itemize}
\end{frame}

\begin{frame}{Cherry-Picking}
  \begin{itemize}
    \item Take a commit from another branch and apply it to yours as well
      \begin{shell}
        \$ git cherry-pick <SHA>
      \end{shell}
    \item Pre-condition is that you did not change same lines
    \item Git keeps track of commits by SHA and can ignore double commits
  \end{itemize}
\end{frame}

\begin{frame}{Other Interesting Commands}
	  Append some changes to the last commit (use only if not pushed):
    \begin{shell}
      \$ git commit --amend \\
    \end{shell}
		Select only some of the changes to a file for a commit:
    \begin{shell}
			\$ git add --patch/-p \\
    \end{shell}
		Graphical tool to select changes to include in a commit:
    \begin{shell}
			\$ git gui \\
    \end{shell}
		Rewrite the history: reorder commits, combine them, \ldots:
    \begin{shell}
			\$ git rebase -i \\
    \end{shell}
\end{frame}

%\section{Beyond Source Control}

\end{document}

% vim:sw=2
